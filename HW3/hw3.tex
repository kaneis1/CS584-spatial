\documentclass[12pt,letterpaper]{article}
\usepackage{forest}
\title{HW3}
\author{ZONGQI CUI}

\begin{document}

\maketitle

\section{}
    \begin{itemize}
        \item[(a)]:
        original tree:\\
            \begin{forest}
                for tree={%
                draw, % draw the tree
                rounded corners, % rounded corners for nodes
                edge={-}, % edge style
                node options={align=center,anchor=north}, % center nodes
                },
                % Tree Structure
                [{13,26, 67}
                    [{4, 7, 9}] 
                    [{14, 15, 16}] 
                    [{28, 41, 56}] 
                    [{68, 72}] 
                ]
            \end{forest}\\
        insert 25:\\
            \begin{itemize}
                \item[1.]find the leaf node that 25 should be inserted into.
                \item[2.]insert 25 into the leaf node.\\
                \par
                \begin{forest}
                    for tree={%
                    draw, % draw the tree
                    rounded corners, % rounded corners for nodes
                    edge={-}, % edge style
                    node options={align=center,anchor=north}, % center nodes
                    },
                    % Tree Structure
                    [{13,26, 67}
                        [{4, 7, 9}] 
                        [{14, 15, 16, 25}]
                        [{28, 41, 56}]
                        [{68, 72}] 
                    ]
                \end{forest}\\ 
                \item[3.]
                    because m=2, adding 25 to the leaf node don't violate the B-tree property.\\  
            \end{itemize}
        insert 17:\\
            \begin{itemize}
                \item[1.]find the leaf node that 17 should be inserted into.
                \item[2.]insert 17 into the leaf node.\\
                \par
                \begin{forest}
                    for tree={%
                    draw, % draw the tree
                    rounded corners, % rounded corners for nodes
                    edge={-}, % edge style
                    node options={align=center,anchor=north}, % center nodes
                    },
                    % Tree Structure
                    [{13,26, 67}
                        [{4, 7, 9}] 
                        [{14, 15, 16, 17, 25}*] 
                        [{28, 41, 56}] 
                        [{68, 72}] 
                    ]
                \end{forest}\\ 
                \item[3.]
                    because m=2, adding 17 to the leaf node violate the B-tree property.\\
                \item[4.]
                    find the mid number of the leaf node, which is 16.\\
                    put 16 to the parent node, and split the leaf node into two nodes.\\
                    \par
                    \begin{forest}
                        for tree={%
                        draw, % draw the tree
                        rounded corners, % rounded corners for nodes
                        edge={-}, % edge style
                        node options={align=center,anchor=north}, % center nodes
                        },
                        % Tree Structure
                        [{13, 16, 26, 67}
                            [{4, 7, 9}] 
                            [{14, 15}] 
                            [{17, 25}]
                            [{28, 41, 56}] 
                            [{68, 72}] 
                        ]
                    \end{forest}\\
                \item[5.] parent node has 4 children, which don't violate the B-tree property.\\
                  
            \end{itemize}
            insert 42:\\
            \begin{itemize}
                \item[1.]find the leaf node that 42 should be inserted into.
                \item[2.]insert 42 into the leaf node.\\
                \par
                \begin{forest}
                    for tree={%
                    draw, % draw the tree
                    rounded corners, % rounded corners for nodes
                    edge={-}, % edge style
                    node options={align=center,anchor=north}, % center nodes
                    },
                    % Tree Structure
                    [{13, 16, 26, 67}
                        [{4, 7, 9}] 
                        [{14, 15}] 
                        [{17, 25}]
                        [{28, 41, 42, 56}] 
                        [{68, 72}] 
                    ]
                \end{forest}\\ 
                \item[3.]
                    adding 42 to the leaf node don't violate the B-tree property.\\
            \end{itemize}
            insert 29:\\
                    \begin{itemize}
                        \item[1.]find the leaf node that 29 should be inserted into.
                        \item[2.]insert 29 into the leaf node.\\
                        \par
                        \begin{forest}
                            for tree={%
                            draw, % draw the tree
                            rounded corners, % rounded corners for nodes
                            edge={-}, % edge style
                            node options={align=center,anchor=north}, % center nodes
                            },
                            % Tree Structure
                            [{13, 16, 26, 67}
                                [{4, 7, 9}] 
                                [{14, 15}] 
                                [{17, 25}]
                                [{28, 29, 41, 42, 56}*] 
                                [{68, 72}] 
                            ]
                        \end{forest}\\ 
                        \item[3.]
                            adding 29 to the leaf node violate the B-tree property.\\
                        \item[4.]
                            find the mid number of the leaf node, which is 41.\\
                            put 41 to the parent node, and split the leaf node into two nodes.\\
                            \par
                            \begin{forest}
                                for tree={%
                                draw, % draw the tree
                                rounded corners, % rounded corners for nodes
                                edge={-}, % edge style
                                node options={align=center,anchor=north}, % center nodes
                                },
                                % Tree Structure
                                [{13, 16, 26, 41, 67}
                                    [{4, 7, 9}] 
                                    [{14, 15}] 
                                    [{17, 25}]
                                    [{28, 29}]
                                    [{42, 56}] 
                                    [{68, 72}] 
                                ]
                            \end{forest}\\
                        \item[5.] parent node has 5 children, which violate the B-tree property.\\
                        \item[6.] find the mid number of the parent node, which is 26.\\
                            put 26 to the grandparent node, and split the parent node into two nodes.\\
                            \par
                            \begin{forest}
                                for tree={%
                                draw, % draw the tree
                                rounded corners, % rounded corners for nodes
                                edge={-}, % edge style
                                node options={align=center,anchor=north}, % center nodes
                                },
                                % Tree Structure
                                [{26}
                                    [{13, 16}
                                        [{4, 7, 9}] 
                                        [{14, 15}] 
                                        [{17, 25}]
                                    ]
                                    [{41, 67}
                                        [{28, 29}]
                                        [{42, 56}] 
                                        [{68, 72}] 
                                    ]
                                ]
                            \end{forest}\\
                       \end{itemize}     
        \item[(b)]:    
            original tree:\\
            \begin{forest}
                for tree={%
                draw, % draw the tree
                rounded corners, % rounded corners for nodes
                edge={-}, % edge style
                node options={align=center,anchor=north}, % center nodes
                },
                % Tree Structure
                [{9}
                    [
                        {3,6}
                        [{1,2}]
                        [{4,5}]
                        [{7,8}]
                    ]
                    [
                        {13,16,21}
                        [{10,11,12}]
                        [{14,15}]
                        [{17,18,19,20}]
                        [{22,23}]
                    ]
                ]
            \end{forest}\\      
                delete 18:\\
                    \begin{itemize}
                        \item[1.]find the leaf node that 18 should be deleted from.
                        \item[2.]delete 18 from the leaf node.\\
                        \item[3.]check the number of the leaf node, which is not violating B-tree property.\\
                        \begin{forest}
                            for tree={%
                            draw, % draw the tree
                            rounded corners, % rounded corners for nodes
                            edge={-}, % edge style
                            node options={align=center,anchor=north}, % center nodes
                            },
                            % Tree Structure
                            [{9}
                                [
                                    {3,6}
                                    [{1,2}]
                                    [{4,5}]
                                    [{7,8}]
                                ]
                                [
                                    {13,16,21}
                                    [{10,11,12}]
                                    [{14,15}]
                                    [{17,19,20}]
                                    [{22,23}]
                                ]
                            ]
                        \end{forest}\\  
                    \end{itemize}
                delete 14:\\
                    \begin{itemize}
                        \item[1.]find the leaf node that 14 should be deleted from which is a leaf node.
                        \item[2.]delete 14 from the leaf node.\\
                        \item[3.]check the number of the leaf node, which is violating B-tree property.\\
                        \begin{forest}
                            for tree={%
                            draw, % draw the tree
                            rounded corners, % rounded corners for nodes
                            edge={-}, % edge style
                            node options={align=center,anchor=north}, % center nodes
                            },
                            % Tree Structure
                            [{9}
                                [
                                    {3,6}
                                    [{1,2}]
                                    [{4,5}]
                                    [{7,8}]
                                ]
                                [
                                    {13,16,21}
                                    [{10,11,12}]
                                    [{15}*]
                                    [{17,19,20}]
                                    [{22,23}]
                                ]
                            ]
                        \end{forest}\\  
                        \item[4.]current node's rightsibling has 3 numbers, which is more than m. rightsibling can let closest 17 be their joint father and previous father 16 come to the current node.\\
                        \begin{forest}
                            for tree={%
                            draw, % draw the tree
                            rounded corners, % rounded corners for nodes
                            edge={-}, % edge style
                            node options={align=center,anchor=north}, % center nodes
                            },
                            % Tree Structure
                            [{9}
                                [
                                    {3,6}
                                    [{1,2}]
                                    [{4,5}]
                                    [{7,8}]
                                ]
                                [
                                    {13,17,21}
                                    [{10,11,12}]
                                    [{15,16}]
                                    [{19,20}]
                                    [{22,23}]
                                ]
                            ]
                        \end{forest}\\
                    \end{itemize}
                 delete 21:
                    \begin{itemize}
                        \item[1.]find the leaf node that 21 should be deleted from which is a directory node.
                        \item[2.]switch 21 with the closest number in the leaf node which is 22\\
                        \item[3.]delete 21 from the leaf node.\\
                        \begin{forest}
                            for tree={%
                            draw, % draw the tree
                            rounded corners, % rounded corners for nodes
                            edge={-}, % edge style
                            node options={align=center,anchor=north}, % center nodes
                            },
                            % Tree Structure
                            [{9}
                                [
                                    {3,6}
                                    [{1,2}]
                                    [{4,5}]
                                    [{7,8}]
                                ]
                                [
                                    {13,17,22}
                                    [{10,11,12}]
                                    [{15,16}]
                                    [{19,20}]
                                    [{23}*]
                                ]
                            ]
                        \end{forest}\\  
                        \item[4.]check the number of the current node, which is violating B-tree property. Since there is no rightsibling we can find leftsibling and find that leftsibling only have 2 numbers. We collapse currentNode and leftsibling and then build new node contaning all keys links to the parent node.\\
                        \begin{forest}
                            for tree={%
                            draw, % draw the tree
                            rounded corners, % rounded corners for nodes
                            edge={-}, % edge style
                            node options={align=center,anchor=north}, % center nodes
                            },
                            % Tree Structure
                            [{9}
                                [
                                    {3,6}
                                    [{1,2}]
                                    [{4,5}]
                                    [{7,8}]
                                ]
                                [
                                    {13,17}
                                    [{10,11,12}]
                                    [{15,16}]
                                    [{19,20,22,23}]
                                ]
                            ]
                        \end{forest}\\
                    \end{itemize}
            delete 1:
                \begin{itemize}
                    \item[1.]find the leaf node that 1 should be deleted from which is a leaf node.
                    \item[2.]delete 1 from the leaf node.\\
                    \begin{forest}
                        for tree={%
                        draw, % draw the tree
                        rounded corners, % rounded corners for nodes
                        edge={-}, % edge style
                        node options={align=center,anchor=north}, % center nodes
                        },
                        % Tree Structure
                        [{9}
                            [
                                {3,6}
                                [{2}*]
                                [{4,5}]
                                [{7,8}]
                            ]
                            [
                                {13,17}
                                [{10,11,12}]
                                [{15,16}]
                                [{19,20,22,23}]
                            ]
                        ]
                    \end{forest}\\  
                    \item[3.]check the number of the current node, which is violating B-tree property. We can find rightsibling that leftsibling only have 2 keys. We collapse currentNode and rightsibling and then build new node contaning all keys links to the parent node.\\
                    \begin{forest}
                        for tree={%
                        draw, % draw the tree
                        rounded corners, % rounded corners for nodes
                        edge={-}, % edge style
                        node options={align=center,anchor=north}, % center nodes
                        },
                        % Tree Structure
                        [{9}
                            [
                                {6}*
                                [{2,3,4,5}]
                                [{7,8}]
                            ]
                            [
                                {13,17}
                                [{10,11,12}]
                                [{15,16}]
                                [{19,20,22,23}]
                            ]
                        ]
                    \end{forest}\\
                    \item[4.]check parent node and find that it don't meet B-tree requirements. We find that parent node's rightsibling only have 2 keys so we collapse them and build a new node\\
                    \begin{forest}
                        for tree={%
                        draw, % draw the tree
                        rounded corners, % rounded corners for nodes
                        edge={-}, % edge style
                        node options={align=center,anchor=north}, % center nodes
                        },
                        % Tree Structure
                        [{6,9,13,17}
                            [{2,3,4,5}]
                            [{7,8}]
                            [{10,11,12}]
                            [{15,16}]
                            [{19,20,22,23}] 
                        ]
                    \end{forest}\\
                \end{itemize}
           
    \end{itemize}
\end{document}
